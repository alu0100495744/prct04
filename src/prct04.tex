\documentclass[a4paper,11pt,twoside]{article}
\usepackage[spanish]{babel}
\usepackage[utf8]{inputenc}
\usepackage{graphicx}

\title{Manejo básico de \LaTeX}
\author{Javier Hernández Pérez}
\date{\today}
\begin{document}
 \maketitle
 %\newpage
 \begin{center}
\section*{Resumen}
Este artículo trata de mostrar que \LaTeX es efectivo.
\end {center}

\section{Motivación y objetivos}
\LaTeX es un gran procesdor de textos con grandes ventajas respecto a otros procesadores de textos como word. Por ejemplo:
\begin{enumerate}
\item \LaTeX es software libre. 
\item No te tienes que preocupar del formato del archivo. Por ejemplo cuando en word quieres que algo esté al principio de una página y cambias algo en páginas anteriores tienes que preocuparte de que siga ahí. 
\end {enumerate}

\section{Acciones con \LaTeX}
\subsection{Gráficos}
\subsection{Tablas}
Un ejemplo de tabla es la sihuiente tabla de valores de la función $y=x$ \par
\begin{table}
\begin{center}
\begin{tabular}{cc}
$y$ & $x$ 
\\
0 & 0 
\\
1 & 1 
\\
2 & 2
\\
3 & 3 
\\
4 & 4 
\\
5 & 5 
\\
\end{tabular}
\end{center}
\caption{tabla}
\label{tabla}
\end{table}
\subsection{Hacer referencias}
En \LaTeX se pueden poner notas al pie de página por ejemplo \footnote {Hola}.
Se pueden hacer referencias por ejemplo \ref {tabla}.
\end{document}
